\section{Problem Statement}

The thesis is part of a research project concerned with developing an authentication system
that makes use of \acp{PUF} for asserting identities.
Authentication is the act of establishing a persons identity by having them prove
that they are, who they say they are \cite[cp.][p. 398]{Basavala2012}.

In today's electronic access control systems, digital keys are used for authentication.
They are stored in a database and assigned to a physical entity like a person's
fingerprint or a smart card.
This approach can be problematic, as the digital key is not directly bound to
the physical object. If an attacker were to gain access to the key,
they could use it to create a copy of the physical entity
and gain access. \cite[cp.][p. 81]{Gao2020}

A solution to this might be \acp{PUF}, which are physical devices, instead of 
digitally stored keys.
During manufacturing, certain uncontrollable production tolerances lead to
each device differing slightly from every other.
This introduces unique characteristics and a certain variability and randomness into every \ac{PUF}, 
which can be used to create a unique fingerprint of every device without the need for a digital key.
While the variation is measurable, it is considered impractical to create an identical physical copy of 
a \ac{PUF}, because it is deemed impossible to gain full control of the
tolerances and imitate the slight variations of another \ac{PUF}, that were introduced in manufacturing. \cite[cp.][p. 81]{Gao2020}

The physical characteristics of \acp{PUF} have to be measured to be able to infer an identity from them.
This is done using a \ac{CRP}. The challenge is some input given to the \ac{PUF}, which uses it to generate an output (the response),
leveraging its unique characteristics. For the same challenge, each individual \ac{PUF} will therefore generate a different response.
By previously generating and recording specific \acp{CRP} or using neural networks to evaluate a response to a
given challenge, the authenticity of a given \ac{PUF} can be established. \cite[cp.][p. 81]{Gao2020}

This thesis focuses on the infrastructure behind a challenge-response based authentication system using \acp{PUF}.
This includes designing a database for storing the required data, specifying a protocol for
communication between the components, developing a backend for generating and sending challenges to the \ac{PUF}
and evaluating the responses, among other things.
Because of this focus the thesis has been titled:
\begin{quote}\begin{center}\emph{An Infrastructure for a Challenge-Response based Authentication System
using Physical Unclonable Functions}\end{center}\end{quote}

\newpage
\section{Research Questions}

It is hard to define specific research questions for this thesis, as more literature research is necessary
to understand, which components the infrastructure consists of and what the main challenges are going to be.
Looking at the project overall, this thesis should mainly answer the following question, which is closely
related to the problem statement and title:

\begin{quote}
\emph{RQ1: What does the infrastructure for an authentication system using \acp{PUF} for physical access control look like?}
\end{quote}

Extending this overarching question, many different subquestions related to the individual components and their relationship could be asked.
Some of them might be:

\begin{quote}
\emph{RQ1.1: Which components are needed as part of the infrastructure?}\\
\emph{RQ1.2: How should the required data be organized in the database?}\\
\emph{RQ1.3: How is data transferred between components?}
\end{quote}

This is definitely not a final set of questions, but rather a look at what the exact research questions in the final thesis may look like.

As implementing a prototype is also part of the thesis, in addition to the conceptual design, 
a second overarching question is this:

\begin{quote}
\emph{RQ2: How can a functional prototype for the proposed infrastructure concept be implemented?}
\end{quote}

There is another aspect of this thesis, which is the evaluation of existing concepts and research on the topic.
It is not planned to make this a seperate research question, as it is not an end in
itself, but rather a means to answer \emph{RQ1} and \emph{2}.

\newpage
\section{Relevance and Existing Research}

The earliest paper about \acp{PUF} found on Google Scholar was published in 2004 and has been cited over 100 times.
Furthermore, Google Scholar lists over 650 results for papers published in 2022 alone,
that mention \acp{PUF}. This shows that the technology has only been conceptualized quite recently 
and has become incredibly relevant in today's research.

\cite{Suh2007} seems to be the most cited paper on \acp{PUF} and shows how the technology can
be used for authentication of physical devices using integrated circuits.

\cite{Herder2014} is another well received paper, which explains possible applications
for \acp{PUF} in authentication scenarios and gives an overview of the different types
of \acp{PUF}. \cite{Devadas2008} additionally goes into the design of \ac{PUF} based
\ac{RFID} devices and outlines a basic challenge-response infrastructure.

A problem that many of these solutions face, is that the \ac{CRP} are explicitly stored
in a database. An attacker with access to this data would be able to
circumvent the entire authentication system, as they would know both challenge and response.
\cite{Chatterjee2019} and \cite{Yilmaz2018} propose protocols that do not require storing \acp{CRP} explicitly
by verifying the responses using cryptographic methods or neural networks.

The papers mentioned above are only a small set of papers that were found during
literature research thus far, which was mostly focused on the basics of \acp{PUF}.
The bulk of literature research still has to be completed using the methodology outlined in the next section.
\cite{Gao2020} is a review article published in 2020 containing references a wide variety
of other research on the topic, which is going to be examined. Additionally, papers about
other relevant fundamental topics such as database design need to be found and studied.

\newpage
\section{Approach and Methodology}

To achieve the research goal of designing and implementing an infrastructure,
each of the sections are approached differently.

In the fundamentals section, literature research is used to gain insight about the current
state of scientific research and to build the basis for the conceptualization and implementation stages.
As there are many different topics like database design or challenge-response protocols to cover
and the review of existing literature is not the main goal of the thesis, a full systematic literature
review will not be conducted. The approach to literature research is described in the following.

The main sources for finding literature will be the EBSCO Discovery Service, IEEEE Xplore,
the ACM Digital Library, Springer Link and Google Scholar.
This should cover the most important sources for scientific literature in the field.
The first step will be to gain a good understanding of existing \ac{PUF} based authentication
systems and their components.
From there, search terms are developed dynamically depending on the required knowledge.
Titles and abstracts of the most cited and relevant papers are scanned to find the papers
and books providing the most relevant information.
Additionally, if current review articles are found for a given topic,
snowballing might be used to find additional relevant information.

As this thesis is part of a research project developing a full \ac{POC} for a \ac{PUF} based
access control system, some project specific knowledge will be required.
An example for this is understanding the specific protocols of the locking systems, that are used
in the project's \ac{POC}.
As this knowledge is held by experts in the research group and external partners,
expert interviews could be conducted to be able to better apply the general knowledge from
literature research to the specific problem space. 

Once the required knowledge has been gathered, the conceptualization phase begins.
Here, the existing solutions discovered in the research for each part of
the infrastructure have to be evaluated based on the requirements of the project.
An approach to this would be to define evaluation criteria and create a weighted sum model
for each component, but this will get very complex.
Another approach might be to take the most fitting ideas from each existing approach and
combining them into a system which fits the concrete problem found in the research project.
The benefit of this is reduced complexity and more flexibility, but it could lead to
"gut feeling" decisions, which would reduce the scientific rigidity of the thesis. More reading has to be done
about which is the best choice.
While reading, the design science research approach was discovered. \cite{Peffers2007}
It could offer interesting methodologies that could be used in the thesis, but has not yet been
evaluated enough.

The goal for the implementation section of the thesis is to create a prototype.
Therefore, a language capable of rapid prototyping like Python is most likely going
to be selected. Some components of the finished system, like the neural networks and \acp{PUF},
are not part of this thesis. Because of this, they are not going to be part of the prototype implementation and
mock implementations of their interfaces and responses will have to be created.
Unit tests could be implemented to evaluate the finished prototype and to better understand
possible edge cases.

\newpage
\section{Structure}

This section outlines a rough structure for the thesis.
Take into account that this is based on the currently limited knowledge about the topic
and is subject to change.

\begin{enumerate}
    \item \textbf{Introduction}: Contains motivation, problem description, research goals, methodology and structure.
    \item \textbf{Fundamentals}: Will mostly consist of literature research and aims to give an understanding
    of what the current state of scientific research on the following topics is:
    \begin{itemize}
        \item Challenge-Response Protocols
        \item Database solutions for access control systems
        \item Neural Networks in relation to \acp{PUF}\\
        This point will not go into the design of the neural networks itself as it's beyond the scope,
        but rather examine the infrastructural requirements that neural networks used to evaluate \acp{PUF}
        have in terms of data size, computational power and other aspects.
        \item Other Components
    \end{itemize}
    It gives a basis for the conception and implementation sections of the thesis.
    \item \textbf{Conception}: Will contain the conceptualization of the infrastructure, likely including:
    \begin{itemize}
        \item Requirements for the proposed solution
        \item Specification of a Challenge-Response Protocol
        \item The concrete design of a database schema, unifying the requirements
        for challenge-response, neural networks and access control systems in one schema.
        \item Description of other components, like a backend for response evaluation
        \item Orchestration and interaction between components
    \end{itemize}
    \item \textbf{Implementation}: Will go over the implementation of a prototypical infrastructure.
    \item \textbf{Evaluation}: Here, the proposed solution will be analyzed in terms of performance, possible security risks,
    limits and potentials for improvement.
    \item \textbf{Conclusion}
\end{enumerate}

\newpage
\section{Goal and Expected Outcome}

Some personal goals for the thesis are the following:
\begin{enumerate}
    \item Answer the selected research questions
    \item Propose an infrastructure that offers value to the research team
    and can be used in the final \ac{POC} for the system
    \item In addition to the specific project, the proposal should
    be generalizable and offer value to other research projects concerned with similar topics
    \item Be able to demonstrate a working prototype
\end{enumerate}

Achieving all of these goals will only be possible if there is enough high-quality existing literature
available for every topic. Also, good communication with the research team is required.
The timeframe is hard to assess and might be too short to create a fully working prototype implementation.
In this case, the scope might have to be adjusted and only a part of the 
proposal might be realized as a prototype.

\newpage
\section{Timetable for Creation of the Thesis}

The following Gantt chart shows a rough timetable for completing the thesis, broken down
into calendar weeks.
The chart begins in calendar week 35 of 2022, as this week contains the submission date of
the expos\'{e}, which is the 31st of August.
All work done prior to that, such as finding a title and writing the expos\'{e}, is not considered
in this view.
The deadline for submission of the thesis is the 10th of November 2022.

\

\begin{ganttchart}[
    expand chart=\textwidth,
    vgrid,
    hgrid,
    y unit chart = 20,
    ]{35}{45}
\gantttitle{2022}{11}\\
\gantttitle{Aug}{1}
\gantttitle{September}{4}
\gantttitle{October}{4}
\gantttitle{Nov}{2}\\
\gantttitlelist{35,...,45}{1} \\

\ganttbar{Literature Research}{35}{37} \\
\ganttbar{Introduction}{36}{37} \\
\ganttbar{Finalize Methodology}{37}{38} \\
\ganttbar{Fundamentals}{37}{39} \\
\ganttbar{Conduct Interviews(?)}{39}{40} \\
\ganttbar{Conception}{40}{42} \\
\ganttbar{Implementation}{41}{43} \\
\ganttbar{Evaluation}{43}{43} \\
\ganttbar{Conclusion}{44}{44} \\
\ganttbar{Final Touches}{44}{44} \\
\ganttbar{Buffer \& Corrections}{45}{45} \\

\ganttvrule[vrule offset=.4]{Submission of Expos\'{e}}{35}
\ganttvrule[vrule offset=.6, vrule label node/.append style={anchor=north east}]{Deadline Thesis}{45}
\end{ganttchart}
